\documentclass{article}

\usepackage{fontspec}
\setmainfont[Mapping=tex-text]{AR PL UMing TW}
%\setmainfont[Mapping=tex-text]{文泉驛正黑}
\XeTeXlinebreaklocale ``zh''
\XeTeXlinebreakskip = 0pt plus 1pt

\usepackage{amsmath}
\usepackage{multirow}

\def\tac{\textasciicircum}
\title{瑲珩}
\author{王湘叡}

\begin{document}
\maketitle{}

在開始之前,讓我們先考慮漢字的``筆劃數''這個屬性。%
假設我們需要知道``曉''這個字的筆劃數,%
最原始的方式是從日的第一筆劃開始,一筆劃一筆劃地去描繪,直到兀的最後一筆。%
在邊寫的同時邊計數,最後得到共十六劃。%

但另一個方法則是:\\
考慮到\\
\[
  \mbox{``曉''}=\mbox{``日''}+\mbox{``堯''}
\]
如果事先已經知道``日''為四劃及``堯''為十二劃,則只要用加法運算就可得到十六劃了。%
然而問題是:要怎麼才可以事先知道``日''、``堯''的筆劃數呢?%
看起來,似乎還是要一筆劃一筆劃去地去計算。
但我們一樣可以採用類似的步驟。
因為\\
\[
  \mbox{``堯''}=\mbox{``垚''}+\mbox{``兀''}
\]
也就是如果事先已經知道``垚''為九劃和``兀''為三劃,則用加法就可得到十二劃。
而要計算``垚''的筆劃,
考慮到\\
\[
  \mbox{``垚''}=\mbox{``土''}\times 3
\]
也就是,如果事先已經知道``土''為三劃,則只要用乘法就可得到``垚''為九劃。
於是,最後的問題變成:要怎麼才可以事先知道``土''、``兀''和``日''的筆劃數呢?
這沒有其它方法,只能用一筆劃一筆劃去地去計算。

從另一個角度來想:
只要我們用一筆劃一筆劃地去計算``土''、``兀''和``日''的筆劃數,
而不用一筆劃一筆劃地去描繪``曉'',
我們就可以計算出``曉''的筆劃數。

看起來,這個方法未必會比一筆劃一筆劃地去計算``曉''的筆劃數來得快,
(因為要一筆劃一筆劃地去計算``土''、``兀''和``日''的筆劃數,且還要做一些加法或乘法運算)。
表面上是如此,但這是在``只計算一個字的筆劃數''的情況下才是如此。
如果現在的情況是``要計算一堆字的筆劃數'',
考量到很多部件在中文字是時常出現的,如``土''、``日''等,
則可以省下不少工夫。
以上面例子來說,只要知道``土''、``兀''和``日''的筆劃數,
我們不但在計算過程中得到了``垚'',``堯'',``曉''的筆劃數。
只要再加上一些計算,
我們同樣可以算出``昌'',``昍'',``晶'',``晿'',``圭'',``圼'',``晆'',``曉''等字的筆劃數。


如果我們將類似的原理套用在輸入法上:
只要我們事先知道一些基本或不易分割的部件的外碼(字碼),
並在結合部件時,採用一定的方式去組合外碼,就可以算出那個字的外碼。
我們就可以省下大量的功夫,甚至是用電腦來計算。
%``曉''=``日''+``堯''\\
%``堯''=``垚''+``兀''\\
%``垚''=``土''+``土''+``土''\\
%%``土'',``日'',``垚'',``堯'',``曉'',''兀''
%%``土'',``垚'',``堯'',``曉'',``兀''

\section{用辭說明}
\begin{itemize}
\item 首碼\\
字根的第一個碼。以``靣''來說,在倉頡中拆作``一田口'',首碼即為一\\
\item 次碼\\
字根的第二個碼。以``靣''來說,在倉頡中拆作``一田口'',首碼即為田\\
\item 三碼\\
字根的第三個碼。以``靣''來說,在倉頡中拆作``一田口'',首碼即為口田\\
\item 末碼\\
字根的最後一個碼。以``靣''來說,在倉頡中拆作``一田口'',首碼即為口\\
\item 尾碼\\
在倉頡中,有時取碼時,並不是取字根的最後一碼,而是最後中的特徵碼。\\
為了與末碼區分,稱之為尾碼。以``靣''來說,在倉頡中拆作``一田口'',首碼即為田
 
\item 簡碼\\
\item 快碼\\
將較常出現的字以較短的編碼來指定者稱之。
將較常出現的字根以較短的編碼來指定者稱之。
如,

\item 標準編碼\\
根據輸入法規則而得到的編碼。

\item 容錯編碼\\
輸入法為了讓使用者有更好的體驗,
為了預防使用者選用不同的字集,為拆碼見解不同於標準,
所以提供多種編碼。

\item 多碼\\
一個字在同一種輸入法下可以有兩種以上的編碼。不同於容錯編碼的概念。\\
      容錯編碼基本上是不標準的碼,如字形不同。但此指的都是標準編碼。
      如注音,一個字可以有很多種唸法,於是,就有多種編碼。
      只要符合規則的,即為標準編碼。
      多碼與容錯編碼間有模糊地帶。

%\item 重碼\\
%不同的字,卻有相同的編碼,稱為重碼。重碼的比率稱為重碼率。\\
%
\end{itemize}


\section{行列輸入法、大易輸入法}
\subsection{輸入法說明}
行列與大易,除了它們所選用的字根集是不同的,大致上是很相像的。
兩者的規則皆為前三後一,即首碼、次碼、三碼和末碼。

同樣也是以``曉''為例。\\
\begin{tabular}{llll}
字  & 行列碼 & 大易碼\\
日  & P(0\tac) & D\\
土  & R(4\tac) & F\\
垚  & RRR(4\tac4\tac4\tac) & FFF\\
兀  & AS(1-2-) & EQ\\
堯  & RRRS(4\tac4\tac4\tac2-) & FFFQ\\
曉  & PRRS(0\tac4\tac4\tac2-) & DFFQ\\
\end{tabular}

\subsection{遞迴式}
分別使用``$\mbox{字}_{\mbox{行}}$''和``$\mbox{字}_{\mbox{易}}$''來表示一個字的行列碼與大易碼。\\
如$\mbox{``曉''}_{\mbox{行}}$=``PRRS''、$\mbox{``曉''}_{\mbox{易}}$=``DFFQ''。
\begin{subequations}
  \begin{align}
  \mbox{字}_{\mbox{行}} &= 行列碼\\
  \mbox{字}_{\mbox{易}} &= 大易碼\\
  甲 \oplus 乙 &= 取(甲+乙)的前三後一碼
  \end{align}
\end{subequations}

若$(\mbox{丙}=\mbox{甲}+\mbox{乙})$
其遞迴算式為:
\begin{subequations}
  \begin{align}
  \mbox{丙}_{\mbox{行}}&=\mbox{取}_{\mbox{前三後一}}(\mbox{甲}_{\mbox{行}} \oplus \mbox{乙}_{\mbox{行}})\\
  \mbox{丙}_{\mbox{易}}&=\mbox{取}_{\mbox{前三後一}}(\mbox{甲}_{\mbox{易}} \oplus \mbox{乙}_{\mbox{易}})
  \end{align}
\end{subequations}

\subsection{資料格式說明}

\section{嘸蝦米輸入法}
\subsection{輸入法說明}
嘸蝦米亦跟行列與大易很相像。嘸蝦米的規則亦為前三後一。
特別的是,嘸蝦米多了一個補碼規定--若取碼不足兩碼,則要根據最後一筆劃添加補碼。
為此,為嘸蝦米添加一個屬性:
用``嘸''、``.嘸補'' 表示一個字的嘸蝦米碼和補碼,\\
\subsection{遞迴式}
使用``$\mbox{字}_{\mbox{嘸補}}$''來表示一個字的嘸蝦米補碼。
使用``$\mbox{字}_{\mbox{嘸}}$''來表示一個字的嘸蝦米碼。
$\mbox{``垚''}_{\mbox{嘸}}=YYY$\\
$\mbox{``兀''}_{\mbox{嘸補}}=L$\\
則
\begin{subequations}
  \begin{align}
  \mbox{字}_{\mbox{嘸}} &= 嘸蝦米碼\\
  \mbox{字}_{\mbox{嘸補}} &= 嘸蝦米補碼\\
  \mbox{字}_{\mbox{嘸暫}} &= 嘸蝦米暫時碼,即沒有補碼\\
  \mbox{字}_{\mbox{嘸}} &=
      \left\{\begin{array}{ll}
        \mbox{字}_{\mbox{嘸暫}}
           & \mbox{若$\mbox{字}_{\mbox{嘸暫}} \geq $三碼}\\
        \mbox{字}_{\mbox{嘸暫}}+\mbox{字}_{\mbox{嘸補}}
           & \mbox{若$\mbox{丙}_{\mbox{嘸暫}} \leq $二碼}
      \end{array}\right.\\
  甲 \oplus 乙 &= 取(甲+乙)的前三後一碼\\
  \end{align}
\end{subequations}

若$(\mbox{丙}=\mbox{甲}+\mbox{乙})$
其遞迴算式為:
\begin{subequations}
  \begin{align}
  \mbox{丙}_{\mbox{嘸補}}&=\mbox{乙}_{\mbox{嘸補}}\\
  \mbox{丙}_{\mbox{嘸暫}}&=\mbox{甲}_{\mbox{嘸暫}} \oplus \mbox{乙}_{\mbox{嘸暫}}\\
  \end{align}
\end{subequations}

\subsection{資料格式說明}
\subsection{注意事項}
嘸蝦米的數字一到十分別為:e,r,s,f,w,l,c,b,k,j。
不用補碼。
目前尚未實做。

\section{鄭碼輸入法}
\subsection{輸入法說明}
鄭碼的規則有點複雜。首先,鄭碼會定義一些字根,並為每個字根編碼。
到目前為止,還跟其它輸入法類似。不同的是,鄭碼會使用二碼來為字根編碼。
鄭碼的字根大多為二碼字,對一些常用的部件則為一碼。
在少數的情況下,有三碼。
鄭碼會依不同的情況,可以省略一些部件或位碼。
須要計算部件數,並依其個數,來決定規則。
不過,鄭碼會用到的字根最多只有頭兩個和尾兩個,總共最多四個。

\subsection{遞迴式}
使用``$\mbox{字}_{\mbox{鄭串}}$''來表示一個字的字根串列。
如$\mbox{``曉''}_{\mbox{鄭串}}=[\mbox{``日''}, \mbox{``土''}, \mbox{``土''}, \mbox{兀}]$\\

\begin{subequations}
  \begin{align}
    \mbox{字}_{\mbox{鄭串}} &= 表示一個字的鄭碼字根串列。\\
    甲 \oplus 乙 &= 取(甲+乙)的鄭碼字根串列的前二後二。
  \end{align}
\end{subequations}

若$(\mbox{丙}=\mbox{甲}+\mbox{乙})$
其遞迴算式為:
\begin{subequations}
  \begin{align}
  \mbox{丙}_{\mbox{鄭串}} &= \mbox{甲}_{\mbox{鄭串}} \oplus \mbox{乙}_{\mbox{鄭串}} \\
  \end{align}
\end{subequations}

只要為每個字求出其所用到的字根串列,再套用其規則,即可算出鄭碼。
此外,在鄭碼中,區碼的重要性大於位碼,也就是說,在無法全取字根的碼時,會先保留區碼。
如``由''的碼為``KIA'',在遇到``由''這個字根,且只能取一碼時,就取''K''。
且只能取二碼時,就取''KI''。
且只能取三碼時,就取''KIA''。

\begin{tabular}{lcll}
           & 首根碼數 & 規則 & 例字\\
  \multirow{3}{*}{二基根字} & 1 & 首根一碼+末根三碼,若末根只有一碼,則補``VV''\\
  & 2 & 首根二碼+末根二碼\\
  & 3 & 首根三碼+末根一碼\\
  \multirow{3}{*}{三基根字} & 1 & 首根一碼+次末根一碼+末根二碼\\
  & 2 & 首根二碼+次末根一碼+末根一碼\\
  & 3 & 首根三碼+末根一碼\\
  \multirow{4}{*}{四基根字以上} & 1 & 首根一碼+次根一碼+次末根一碼+末根二碼\\
  & 2 & 首根二碼+次末根一碼+末根一碼\\
  & 3 & 首根三碼+末根一碼\\
\end{tabular}

\section{倉頡輸入法}
\subsection{輸入法說明}
倉頡為一複雜的輸入法。會考慮字的結構。

倉頡的規則分為整體字和組合字。
整體字若不足四碼則全取,否則取首、次、三、尾碼。
若為組合字,字首取首、尾兩碼,字身取首、次、尾三碼。

若字身為組合字,當次字首為一碼時,取次字首和次字身取首、尾兩碼。
否則取次字身取首、尾兩碼和次字身取尾碼。

倉頡的分割,是以視覺上的分割,而非邏輯上的分割,如``順''。
對於熟悉中文字的人,會很直覺地分成``川''和``頁''。
但倉頡則是分成``丨''及剩下的部分(即`丨丨頁')。

此外,還要考慮字身的方向性。可分水平、垂直,其它。
如``卲''不分為``召''和``叩'',而是分為``召''和``阝''。
因為``召''的分向為垂直方向,但``卲''的方向為水平。

\subsection{遞迴式}
\begin{subequations}
  \begin{align}
    \mbox{字}_{\mbox{倉串}} &= 表示一個字的倉頡碼字根串列。\\
    \mbox{字}_{\mbox{倉向}} &= 表示一個字的字根組成方向。\\
    \mbox{字}_{\mbox{倉身}} &= 表示一個字當另一個字的字身時的倉頡碼。\\
  \end{align}
\end{subequations}

如$\mbox{``曉''}_{\mbox{倉串}}=[\mbox{``日''}, \mbox{``土''}, \mbox{``土''}, \mbox{``山''}]$\\
如$\mbox{``曉''}_{\mbox{倉向}}=`-'$\\
如$\mbox{``曉''}_{\mbox{倉身}}=[\mbox{``日''}, \mbox{``土''}, \mbox{``山''}]$\\
\begin{subequations}
  \begin{align}
    \oplus_{\mbox{字}}(甲) &=
      \left\{\begin{array}{ll}
        \mbox{甲}_{\mbox{倉串}}
           & \mbox{若$\mbox{甲}_{\mbox{倉向}} = \mbox{字}_{\mbox{倉向}}$}\\
        $[$ \mbox{甲}_{\mbox{倉身}} $]$
           & \mbox{若$\mbox{甲}_{\mbox{倉向}} \neq \mbox{字}_{\mbox{倉向}}$}
      \end{array}\right.\\
  \mbox{丙}_{\mbox{倉串}} &= \oplus_{\mbox{丙}}(甲) + \oplus_{\mbox{丙}}(乙)\\
  \end{align}
\end{subequations}

\subsection{資料格式說明}
倉頡對於字根的取碼有特殊規定。
若字本身為輔助字根,則不能直接取碼。

如``氵''為``水''的輔助字根,``工''為``一''的輔助字助。
在``江''時,其倉頡碼為``水一''
但若``氵''當獨立字時,則要取碼``卜一''
若``工''當獨立字時,則要取碼``一中一''

\subsection{注意事項}
注意:實作時,有些字的特徵碼未處理上確,造成錯誤,有待解決。如``贕''。

\subsection{資料格式說明}
\end{document}
