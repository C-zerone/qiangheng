
\chapter{原理}
在開始之前,讓我們可以利用漢字的``筆劃數''的這個屬性來做說明。%
假設我們需要知道``曉''這個字的筆劃數,%
最直覺得方法就是:從第一筆(日的第一筆)到最後一筆(兀的最後一筆),%
一筆一筆地去描繪,在描繪的同時邊計數,最後得到共十六劃,這樣就完成了任務。%

這個方法雖然直覺而且簡單,%
但若需要算出的不是一``個''字的筆劃數,而是一``羣''字(如所有漢字)的筆劃,這個方法就顯得笨拙。
首先,我們需要花費大量的人力去做這件事,而且過程中極易出錯。
其次,每個人可能會使用不同的標準(如對同一個字選擇不同的字形或筆順)。
此外,如果在過程中對標準做出修訂,則又要重新花費大量的人力做一樣的事。

之所以會如此的原因在於這方法沒有充分使用到漢字系統的特性。
在漢字系統中,大部分的漢字都是由其它漢字或字根所組合出來的。
於是,新的漢字具有字根的特性。

如果考慮了這個特性,我們可以採用另一個方法:\\
因為\\
\[
  \qhchar{``曉''}=\qhchar{``日''}+\qhchar{``堯''}
\]
若已經知道``日''為四劃而``堯''為十二劃,則只要用加法就可得到``曉''為十六劃了。%

然而一個問題是:要怎麼才可以知道``日''、``堯''的筆劃數呢?%
我們一樣可以採用類似的方法。
因為\\
\[
  \qhchar{``堯''}=\qhchar{``垚''}+\qhchar{``兀''}
\]
也就是如果事先已經知道``垚''為九劃和``兀''為三劃,則用加法就可得到十二劃。
同樣地,考慮到\\
\[
  \qhchar{``垚''}=\qhchar{``土''}\times 3
\]
也就是,如果事先已經知道``土''為三劃,則只要用乘法就可得到``垚''為九劃。

於是,最後的問題變成:只要知道``土''、``兀''和``日''的筆劃數,我們就可以算出``曉''的筆劃數。
而要知道``土''、``兀''和``日''的筆劃數,則只能用一筆劃一筆劃去地去計算。

從另一個角度來想:
只要我們用一筆劃一筆劃地去計算``土''、``兀''和``日''的筆劃數,
而不用一筆劃一筆劃地去描繪``曉'',
我們就可以計算出``曉''的筆劃數。

實際上,對求出``曉''的筆劃這個問題而言,
這個方法不會比一筆劃一筆劃地去計算來得快,
(因為要一筆劃一筆劃地去計算``土''、``兀''和``日''的筆劃數,且還要做一些加法或乘法運算)。

如果現在的情況是``要計算一堆字的筆劃數'',則這個方法可以大幅減少工作量及時間。

考量到很多部件在中文字是時常出現的,如``土''、``日''等,
以上面例子來說,只要知道``土''、``兀''和``日''的筆劃數,
我們不但在計算過程中得到了``垚'',``堯'',``曉''的筆劃數。
只要再加上一些計算,
我們同樣可以算出``昌'',``昍'',``晶'',``晿'',``圭'',``圼'',``晆'',``曉''等字的筆劃數。


此外,這個方法也適合自動化。
只要知道每個漢字的組成方式,並且有一些基礎字根的資料。
我們就可以自動算出全部的值。
自動化的好處還有標準統一,如果有一些標準改了,也可快速重新計算。

\begin{subequations}
  \begin{align}
  \qhstroke{字} &= \text{筆劃數}\\
  \qhstroke{丙} &= \qhstroke{甲} + \qhstroke{乙}\
  \end{align}
\end{subequations}

如果我們的目標不是計算``筆劃數'',而是創造字形,這個方法一樣可以適用,這就是``動態組字''的範疇。
如果我們的目標不是計算``筆劃數'',而是輸入碼,這個方法一樣可以適用,這就是本計劃的範疇。

如果我們將類似的原理套用在輸入法上:
只要我們事先知道一些基本或不易分割的部件的外碼(字碼),
並在結合部件時,採用一定的方式去組合外碼,就可以算出那個字的外碼。
我們就可以省下大量的功夫,甚至是用電腦來計算。
%``曉''=``日''+``堯''\\
%``堯''=``垚''+``兀''\\
%``垚''=``土''+``土''+``土''\\
%%``土'',``日'',``垚'',``堯'',``曉'',''兀''
%%``土'',``垚'',``堯'',``曉'',``兀''

本計劃目前選擇了五種字形輸入法:倉頡、行列、嘸蝦米、大易、鄭碼。

\section{用辭說明}
\begin{itemize}
\item 首碼\\
字根的第一個碼。以``靣''來說,在倉頡中拆作``一田口'',首碼即為一\\
\item 次碼\\
字根的第二個碼。以``靣''來說,在倉頡中拆作``一田口'',首碼即為田\\
\item 三碼\\
字根的第三個碼。以``靣''來說,在倉頡中拆作``一田口'',首碼即為口田\\
\item 末碼\\
字根的最後一個碼。以``靣''來說,在倉頡中拆作``一田口'',首碼即為口\\
\item 尾碼\\
在倉頡中,有時取碼時,並不是取字根的最後一碼,而是最後中的特徵碼。\\
為了與末碼區分,稱之為尾碼。以``靣''來說,在倉頡中拆作``一田口'',首碼即為田
 
\item 簡碼\\
\item 快碼\\
將較常出現的字以較短的編碼來指定者稱之。
將較常出現的字根以較短的編碼來指定者稱之。
如,

\item 標準編碼\\
根據輸入法規則而得到的編碼。

\item 容錯編碼\\
輸入法為了讓使用者有更好的體驗,
為了預防使用者選用不同的字集,為拆碼見解不同於標準,
所以提供多種編碼。

\item 多碼\\
一個字在同一種輸入法下可以有兩種以上的編碼。不同於容錯編碼的概念。\\
      容錯編碼基本上是不標準的碼,如字形不同。但此指的都是標準編碼。
      如注音,一個字可以有很多種唸法,於是,就有多種編碼。
      只要符合規則的,即為標準編碼。
      多碼與容錯編碼間有模糊地帶。

%\item 重碼\\
%不同的字,卻有相同的編碼,稱為重碼。重碼的比率稱為重碼率。\\
%
\end{itemize}


\chapter{輸入法}
\section{行列輸入法、大易輸入法}
\subsection{輸入法說明}
行列與大易輸入法,看起來像是兩個截然不同的輸入法(事實上,學習方式也非常地不同)。%
然而,兩者除了所選用的字根及排列方式不同外,
就規則而言,是頗為相像的。
兩者的規則皆為前三後一,
即若一個字拆成字根後,字根個數小於四個則全取。
大於等於四個,則取首碼、次碼、三碼和末碼。

同樣也是以``曉''為例。\\
\begin{tabular}{llll}
字  & 行列碼 & 大易碼\\
日  & P(0\tac) & D\\
土  & R(4\tac) & F\\
垚  & RRR(4\tac4\tac4\tac) & FFF\\
兀  & AS(1-2-) & EQ\\
堯  & RRRS(4\tac4\tac4\tac2-) & FFFQ\\
曉  & PRRS(0\tac4\tac4\tac2-) & DFFQ\\
\end{tabular}

\subsection{遞迴式}
為方便說明起見,分別使用$\QhAr{字}$和$\QhDy{字}$來表示一個字的行列碼與大易碼。\\
如: $\QhAr{曉}$ =``PRRS''、$\QhDy{曉}$=``DFFQ''。
\begin{subequations}
  \begin{align}
  \QhAr{字} &= 行列碼\\
  \QhDy{字} &= 大易碼\\
  \QhArRlist{字} &= 將字拆成行列字根所構成的串列\\
  \QhDyRlist{字} &= 將字拆成大易字根所構成的串列\\
  \end{align}
\end{subequations}

若$(\qhchar{丙}=\qhchar{甲}+\qhchar{乙})$,其遞迴算式分別為:
\begin{subequations}
  \begin{align}
  \QhArRlist{丙}&=\QhArRlist{甲} \oplus \QhArRlist{乙}\\
  \QhDyRlist{丙}&=\QhDyRlist{甲} \oplus \QhDyRlist{乙}
  \end{align}
\end{subequations}

然而考慮到:
\begin{subequations}
  \begin{align}
  \QhArRule{}({丙})&=\QhRuleThreeOne{}({甲})\\
  \QhDyRule{}({丙})&=\QhRuleThreeOne{}({甲})
  \end{align}
\end{subequations}

則可以得到其衍生的遞迴算式為:
\begin{subequations}
  \begin{align}
  \QhAr{丙}&=\QhArRule{}(\QhAr{甲} \oplus \QhAr{乙})\\
  \QhDy{丙}&=\QhDyRule{}(\QhDy{甲} \oplus \QhDy{乙})
  \end{align}
\end{subequations}

\subsection{注意事項}
對於行列輸入法而言,``囚''即為特別,它有兩個拆法。
在``溫''中及在``囚''中的拆法不一。照行列的說法是以面積來決定。

\section{嘸蝦米輸入法}
\subsection{輸入法說明}
嘸蝦米的規則跟行列與大易很相像。同樣為前三後一。
特別的是,嘸蝦米多了一個補碼規定--若取碼不足兩碼,則要根據最後一筆劃添加補碼。
為此,為嘸蝦米添加一個屬性:
用``嘸''、``.嘸補'' 表示一個字的嘸蝦米碼和補碼,\\
\subsection{遞迴式}
使用``$\QhBscomp{字}$''來表示一個字的嘸蝦米補碼。
使用``$\QhBs{字}$''來表示一個字的嘸蝦米碼。
$\QhBs{垚}=YYY$\\
$\QhBscomp{兀}=L$\\
則
\begin{subequations}
  \begin{align}
  \QhBs{字} &= 嘸蝦米碼\\
  \QhBscomp{字} &= 嘸蝦米補碼\\
  \QhBstemp{字} &= 嘸蝦米暫時碼,即沒有補碼\\
  \QhBs{字} &=
      \left\{\begin{array}{ll}
        \QhBstemp{字}
           & \text{若$\QhBstemp{字} \geq $三碼}\\
        \QhBstemp{字}+\QhBscomp{字}
           & \text{若$\QhBstemp{丙} \leq $二碼}
      \end{array}\right.\\
  甲 \oplus 乙 &= 取(甲+乙)的前三後一碼\\
  \QhBsRlist{字} &= 將字拆成嘸蝦米字根所構成的串列\\
  \end{align}
\end{subequations}

若$(\qhchar{丙}=\qhchar{甲}+\qhchar{乙})$,其遞迴算式為:
\begin{subequations}
  \begin{align}
  \QhBsRlist{丙}&=\QhBsRlist{甲} \oplus \QhBsRlist{乙}\\
  \end{align}
\end{subequations}

考慮到:
\begin{subequations}
  \begin{align}
  \QhBsRule{}({字})&=\QhRuleThreeOne{}({字})
  \end{align}
\end{subequations}

則可以得到其衍生的遞迴算式為:
\begin{subequations}
  \begin{align}
  \QhBs{丙}&=\QhBsRule{}(\QhBs{甲} \oplus \QhBs{乙})\\
  \end{align}
\end{subequations}

\subsection{注意事項}

\section{鄭碼輸入法}
\subsection{輸入法說明}
鄭碼的規則有點複雜。首先,鄭碼會定義一些字根,並為每個字根編碼。
到目前為止,還跟其它輸入法類似。
不同的是,為字根編碼時,其它輸入法都是使用單碼,但鄭碼會使用雙碼。
不過對一些常用的字根,則會優化為一碼。
在極少數的情況下,則用三碼來為字根編碼。
在將一個字拆為一個個字根的序列時,鄭碼會依據字根數,來決定其編碼規則。
不過,鄭碼會用到的字根最多只有頭兩個和尾兩個,總共最多四個。

\subsection{遞迴式}
使用``$\QhZmRlist{字}$''來表示一個字的字根串列。
如:$\QhZmRlist{曉}=[\qhchar{``日''}, \qhchar{``土''}, \qhchar{``土''}, \qhchar{``土''}, \qhchar{``兀''}]$。\\
鄭碼的計算方式為:
\begin{subequations}
  \begin{align}
    \QhZmRlist{字} &= 表示一個字的鄭碼字根串列。\\
    \QhZm{字} &= \QhZmRule(\QhZmRlist{字})\\
  \end{align}
\end{subequations}

若$(\qhchar{丙}=\qhchar{甲}+\qhchar{乙})$,其遞迴算式為:
\begin{subequations}
  \begin{align}
  \QhZmRlist{丙}&=\QhZmRlist{甲} \oplus \QhZmRlist{乙}\\
  \end{align}
\end{subequations}

然而,由於鄭碼的計算方式較為複雜。%
無法將鄭碼字根串列的遞迴和鄭碼的計算方式結合成一個簡單的遞迴式。%
以下列出鄭碼的計算規則:\\
\begin{tabular}{lcll}
           & 首根碼數 & 規則 & 例字\\
  \multirow{3}{*}{二基根字} & 1 & 首根一碼+末根三碼,若末根只有一碼,則補``VV''\\
  & 2 & 首根二碼+末根二碼\\
  & 3 & 首根三碼+末根一碼\\
  \multirow{3}{*}{三基根字} & 1 & 首根一碼+次末根一碼+末根二碼\\
  & 2 & 首根二碼+次末根一碼+末根一碼\\
  & 3 & 首根三碼+末根一碼\\
  \multirow{4}{*}{四基根字以上} & 1 & 首根一碼+次根一碼+次末根一碼+末根二碼\\
  & 2 & 首根二碼+次末根一碼+末根一碼\\
  & 3 & 首根三碼+末根一碼\\
\end{tabular}

\section{倉頡輸入法}
\subsection{輸入法說明}
倉頡輸入法算是最為複雜的一個輸入法了。%
主要是倉頡輸入法不只考慮字根,還會考慮字的結構。

倉頡的規則分為整體字和組合字。
整體字若不足四碼則全取,否則取首、次、三、尾碼。
若為組合字,字首取首、尾兩碼,字身取首、次、尾三碼。

若字身為組合字,當次字首為一碼時,取次字首和次字身取首、尾兩碼。
否則取次字身取首、尾兩碼和次字身取尾碼。

倉頡的分割,是以視覺上的分割,而非邏輯上的分割,如``順''。
對於熟悉中文字的人,會很直覺地分成``川''和``頁''。
但倉頡則是分成``丨''及剩下的部分(即`丨丨頁')。

此外,還要考慮字身的方向性。可分水平、垂直,其它。
如``卲''不分為``刀''和``叩'',而是分為``召''和``阝''。
因為``召''的分向為垂直方向,但``卲''的方向為水平。

\subsection{遞迴式}
\begin{subequations}
  \begin{align}
    \QhCjlist{字} &= 表示一個字的倉頡碼字根串列。\\
    \QhCjdir{字} &= 表示一個字的字根組成方向。\\
    \QhCjbody{字} &= 表示一個字當另一個字的字身時的倉頡碼。\\
  \end{align}
\end{subequations}

如$\QhCjlist{``曉''}=[\qhchar{``日''}, \qhchar{``土''}, \qhchar{``土''}, \qhchar{``山''}]$\\
如$\QhCjdir{``曉''}=`-'$\\
如$\QhCjbody{``曉''}=[\qhchar{``日''}, \qhchar{``土''}, \qhchar{``山''}]$\\

\begin{subequations}
  \begin{align}
    \QhCjmerge{字}{甲} &=
      \left\{\begin{array}{ll}
        \QhCjlist{甲}
           & \text{若$\QhCjdir{甲} = \QhCjdir{字}$}\\
        $[$ \QhCjbody{甲} $]$
           & \text{若$\QhCjbody{甲} \neq \QhCjbody{字}$}
      \end{array}\right.\\
  \QhCjbody{丙} &= \QhCjmerge{丙}{甲} + \QhCjmerge{丙}{乙}\\
  \QhCjdir{丙} &= 由丙的組成方式計算出方向。\\
  \end{align}
\end{subequations}

\subsection{注意事項}
倉頡對於字根的取碼有特殊規定。
若字本身為輔助字根,則不能直接取碼。

%如``氵''為``水''的輔助字根,``工''為``一''的輔助字助。
%在``江''時,其倉頡碼為``水一''
%但若``氵''當獨立字時,則要取碼``卜一''
%若``工''當獨立字時,則要取碼``一中一''

\section{四角號碼檢字法}
\subsection{輸入法說明}
請參見 http://zh.wikipedia.org/wiki/四角號碼 。

\section{中國字庋㩪法}
\subsection{輸入法說明}
請參見 http://zh.wikipedia.org/wiki/中國字庋㩪法 。

