\documentclass{article}

\usepackage{fontspec}
\setmainfont[Mapping=tex-text]{AR PL UMing TW}
%\setmainfont[Mapping=tex-text]{文泉驛正黑}
\XeTeXlinebreaklocale ``zh''
\XeTeXlinebreakskip = 0pt plus 1pt

\usepackage{amsmath}

\def\tac{\textasciicircum}
\title{瑲珩}
\author{王湘叡}

\begin{document}
\maketitle{}

在開始之前,讓我們先考慮漢字的``筆劃數''這個屬性。%
假設要計算``曉''這個字的筆劃數,%
一個方式是一筆劃一筆劃地去寫出這個字,同時計數,最後得到十六劃。%
但另一個方法則是:\\
考慮到\\
\[
  \mbox{``曉''}=\mbox{``日''}+\mbox{``堯''}
\]
如果事先已經知道``日''為四劃及``堯''為十二劃,則只要用運算就可得到十六劃。
然而,問題是:要怎麼才可以事先知道``日''、``堯''的筆劃數呢?
看起來,似乎還是要一筆劃一筆劃去地去計算。
但我們一樣可以使用類似的方式。
考慮到\\
\[
  \mbox{``堯''}=\mbox{``垚''}+\mbox{``兀''}
\]
也就是如果事先已經知道``垚''為九劃和``兀''為三劃,則要用加法就可得到十二劃。
於是,問題變成:要怎麼才可以事先知道``垚''、``兀''的筆劃數呢?
若要計算``垚''、``兀''的筆劃,一樣要一筆一筆去算
對於``垚'',考慮到\\
\[
  \mbox{``垚''}=\mbox{``土''}\times 3
\]
也就是,如果事先已經知道``土''為三劃,則只要用乘法就可得到``垚''為九劃。
於是,最後的問題變成:要怎麼才可以事先知道``土''、``兀''和``日''的筆劃數呢?
這沒有其它方法,只能用一筆劃一筆劃去地去計算。

也就是說:只要我們用一筆劃一筆劃地去計算``土''、``兀''和``日''的筆劃數,
我們就可以計算出``曉''的筆劃。
看起來,這個方法未必會比一筆劃一筆劃地去計算``曉''的筆劃數來得快,
(因為還是要一筆劃一筆劃地去計算``土''、``兀''和``日''的筆劃數,且還要做一些加法或乘法運算)。
表面上是如此,但這是在``只計算一個字的筆劃數''的情況下才是如此。
如果現在的情況是``要計算一堆字的筆劃數'',
考量到很多部件在中文字是時常出現的,如``土''、``日''等,
也就是,如果只要能事先算出來,就可以省下很多功夫。

如果我們將類似的原理套用在輸入法上:
只要我們事先計算一些基本或不易分割的部件的外碼(字碼),
並在結合部件時,採用一定的方式去組合外碼,就可以算出那個字的外碼。
我們就可以省下大量的功夫,甚至是用電腦來計算。
%``曉''=``日''+``堯''\\
%``堯''=``垚''+``兀''\\
%``垚''=``土''+``土''+``土''\\
%%``土'',``日'',``垚'',``堯'',``曉'',''兀''
%%``土'',``垚'',``堯'',``曉'',``兀''

\section{行列、大易}
行列、大易的規則皆為前三後一。

同樣也是以``曉''為例。
\begin{tabular}{llll}
字  & 行列碼 & 大易碼\\
日  & P(0\tac) & D\\
土  & R(4\tac) & F\\
垚  & RRR(4\tac4\tac4\tac) & FFF\\
兀  & AS(1-2-) & EQ\\
堯  & RRRS(4\tac4\tac4\tac2-) & FFFQ\\
曉  & PRRS(0\tac4\tac4\tac2-) & DFFQ\\
\end{tabular}

如果用``.行''和``.易''來表示一個字的行列碼和大易碼,\\
$\mbox{``垚''}_{\mbox{行}}=PPP$\\
$\mbox{``曉''}_{\mbox{易}}=QFFD$\\
且 $(\mbox{丙}=\mbox{甲}+\mbox{乙})$
我們可以用
\begin{subequations}
  \begin{align}
  \mbox{丙}_{\mbox{行}}&=\mbox{取前三後一}(\mbox{甲}_{\mbox{行}}+\mbox{乙}_{\mbox{行}})\\
  \mbox{丙}_{\mbox{易}}&=\mbox{取前三後一}(\mbox{甲}_{\mbox{易}}+\mbox{乙}_{\mbox{易}})
  \end{align}
\end{subequations}
來計算

\section{嘸蝦米}
嘸蝦米的規則亦為前三後一,但不同的是,嘸蝦米多了一個補碼規定
--若取碼不足兩碼,則要根據最後一筆劃添加補碼。
為此,為嘸蝦米添加一個屬性:
用``嘸''、``.嘸補'' 表示一個字的嘸蝦米碼和補碼,\\
$\mbox{``垚''}_{\mbox{嘸}}=YYY$\\
$\mbox{``兀''}_{\mbox{嘸補}}=L$\\
函數``\mbox{補}(X, Y)'' 則會根據 $X$ 的長度是否小於等於二碼來決定是否附加 $Y$
則
\begin{subequations}
  \begin{align}
  \mbox{丙}_{\mbox{嘸補}}&=\mbox{乙}_{\mbox{嘸補}}\\
  \mbox{丙}_{\mbox{嘸}}&=\mbox{補}(\mbox{取前三後一}(\mbox{甲}_{\mbox{嘸}}+\mbox{乙}_{\mbox{嘸}}),
                          \mbox{丙}_{\mbox{嘸補}})
  \end{align}
\end{subequations}
\section{倉頡}
倉頡的規則分為整體字和組合字。
整體字若不足四碼則全取,否則取首、二、三、末碼。
若為組合字,
``倉本''表示當成本字的倉頡碼。
``倉部''表示當成部件的倉頡碼。\\
$\mbox{``垚''}_{\mbox{嘸}}=GGG$\\
$\mbox{``八''}_{\mbox{倉部}}=HO$\\

\begin{subequations}
  \begin{align}
  \mbox{丙}_{\mbox{倉本}}&=\mbox{首尾}(\mbox{甲}_{\mbox{倉部}})
      + \mbox{首二尾}({\mbox{乙}_{\mbox{倉部}}})\\
  \mbox{丙}_{\mbox{倉部}}&= \mbox{首二尾}(
      \mbox{首尾}(\mbox{甲}_{\mbox{倉部}})
      + \mbox{首尾}(\mbox{甲}_{\mbox{倉部}})
      )
  \end{align}
\end{subequations}

\section{鄭碼}
鄭碼的規則
鄭碼的部件大多為二碼字,對一些常用的部件則為一碼
在少數的情況下,有三碼。
首碼稱為區碼,其餘稱為位碼。並依不同的情況,可以省略一些部件或位碼。
須要計算部件數,並依其個數,來決定規則。

\begin{tabular}{llll}
           & 首根 & 規則 & 例字\\
  二基根字 & \\
  三基根字 & \\
  四基根字 & \\
\end{tabular}
``鄭''表示一個字的鄭碼。
``鄭根串''表示一個字的字根串列。
``鄭根數''表示一個字可以拆成幾個字個。
``鄭首''表示一個字首個字根的碼數。

$\mbox{``堯''}_{\mbox{鄭根串}}=[\mbox{``土''}, \mbox{``土''}, \mbox{``土''}, \mbox{兀}]$\\
$\mbox{``堯''}_{\mbox{鄭根數}}=4$\\
$\mbox{``堯''}_{\mbox{鄭根首}}=1$($\mbox{``鄭''}_{``土''}=\mbox{B}$)\\
$\mbox{``兀''}_{\mbox{鄭根首}}=2$($\mbox{``鄭''}_{``兀''}=\mbox{GR}$)\\

\begin{subequations}
  \begin{align}
  \mbox{丙}_{\mbox{鄭根串}} &= \mbox{甲}_{\mbox{鄭根串}} + \mbox{乙}_{\mbox{鄭根串}} \\
%  \mbox{丙}_{\mbox{鄭根數}} &= \max(4, \mbox{甲}_{\mbox{鄭根}} + \mbox{乙}_{\mbox{鄭根}} )\\
  \mbox{丙}_{\mbox{鄭根數}} &= \max(4, \mbox{丙}_{\mbox{鄭根數}} )\\
  \mbox{丙}_{\mbox{鄭首}} &= \mbox{甲}_{\mbox{鄭首}}
  \end{align}
\end{subequations}

\begin{subequations}
  \begin{align}
  \mbox{丙}_{\mbox{鄭}} &= \mbox{丙}_{\mbox{鄭根串}}[0][0]\\
      & + \mbox{丙}_{\mbox{鄭根串}}[-1][0:3] if \mbox{丙}_{\mbox{鄭根數}}=2 and \mbox{丙}_{\mbox{鄭首}}=1 \\
  \mbox{丙}_{\mbox{鄭}} &= \mbox{丙}_{\mbox{鄭根串}}[0][0:2]\\
      & + \mbox{丙}_{\mbox{鄭根串}}[-1][0:2] if \mbox{丙}_{\mbox{鄭根數}}=2 and \mbox{丙}_{\mbox{鄭首}}=2 \\
  \mbox{丙}_{\mbox{鄭}} &= \mbox{丙}_{\mbox{鄭根串}}[0][0:3]\\
      & + \mbox{丙}_{\mbox{鄭根串}}[-1][0] if \mbox{丙}_{\mbox{鄭根數}}=2 and \mbox{丙}_{\mbox{鄭首}}=3 \\
  \mbox{丙}_{\mbox{鄭}} &= \mbox{丙}_{\mbox{鄭根串}}[0][0]\\
      & + \mbox{丙}_{\mbox{鄭根串}}[-2][-1]\\
      & + \max(2, \mbox{丙}_{\mbox{鄭根串}}[-1][:]) if \mbox{丙}_{\mbox{鄭根數}}=3 and \mbox{丙}_{\mbox{鄭首}}=1 \\
  \mbox{丙}_{\mbox{鄭}} &= \mbox{丙}_{\mbox{鄭根串}}[0][0:2]\\
      & + \mbox{丙}_{\mbox{鄭根串}}[-2][-1]\\
      & + \mbox{丙}_{\mbox{鄭根串}}[-1][-1] if \mbox{丙}_{\mbox{鄭根數}}=3 and \mbox{丙}_{\mbox{鄭首}}=2 \\
  \mbox{丙}_{\mbox{鄭}} &= \mbox{丙}_{\mbox{鄭根串}}[0][0:3]+\mbox{丙}_{\mbox{鄭根串}}[-1][0]
      if \mbox{丙}_{\mbox{鄭根數}}=3 and \mbox{丙}_{\mbox{鄭首}}=3\\
  \mbox{丙}_{\mbox{鄭}} &= \mbox{丙}_{\mbox{鄭根串}}[0][0]\\
      & + \mbox{丙}_{\mbox{鄭根串}}[2][0]\\
      & + \mbox{丙}_{\mbox{鄭根串}}[-2][0]\\
      & + \mbox{丙}_{\mbox{鄭根串}}[-1][0] if \mbox{丙}_{\mbox{鄭根數}}=4 and \mbox{丙}_{\mbox{鄭首}}=1\\
  \mbox{丙}_{\mbox{鄭}} &= \mbox{丙}_{\mbox{鄭根串}}[0][0:2]\\
      & + \mbox{丙}_{\mbox{鄭根串}}[-2][0]\\
      & + \mbox{丙}_{\mbox{鄭根串}}[-1][0] if \mbox{丙}_{\mbox{鄭根數}}=4 and \mbox{丙}_{\mbox{鄭首}}=2\\
  \mbox{丙}_{\mbox{鄭}} &= \mbox{丙}_{\mbox{鄭根串}}[0][0:3]+\mbox{丙}_{\mbox{鄭根串}}[-1][0]
      if \mbox{丙}_{\mbox{鄭根數}}=4 and \mbox{丙}_{\mbox{鄭首}}=3\\
  \end{align}
\end{subequations}
\end{document}
