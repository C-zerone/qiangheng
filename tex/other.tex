\chapter{其它事項}
字頻資訊放於 qhdata/frequency/CJK.xml 。
這是由 SCIM 的廣東拼音字碼表所擷取出來的。


也可以根據一個編碼加以分類。
\begin{enumerate}
\item[一、]字符碼\\
當此部件為一字符時所用的碼。
對一些輸入法而言,又稱作``簡快碼''。
如:嘸蝦米的`一'編碼為 E 但為簡碼,倉頡的`冖‘編碼為 戈弓(IN) 。
\item[二、]字根碼\\
當此部件為一字根時所用的碼。
對一些輸入法而言,又稱作``簡字根''。
如:嘸蝦米的`一'編碼為 E 且補碼為 E,倉頡的`冖‘編碼為 月(B) 。
\end{enumerate}

也可以根據一個編碼所產生的方式來加以分類:
\begin{enumerate}
\item[一、]標準碼\\
依據輸入法標準而生的碼,且一字一碼。
\item[二、]簡快碼\\
為了加快輸入速度,可能會依字頻而給予較簡編碼產生的碼。 
\item[三、]容錯碼\\
使用者可能犯錯而產生的碼,如選用錯誤字形,寫錯字。
\end{enumerate}

排列組合就有:
\begin{enumerate}
\item[一、]標準字符碼與標準字根碼\\
依據輸入法標準,當一個部件是字符或是字根時,會有的編碼。
\item[二、]簡快字符碼與簡快字根碼\\
一個字符,如果字頻高,輸入法給予其較短的編碼,即為簡快字符碼。
一個字根,如果由其組成的字很多高頻字,輸入法給予其較短的編碼,即為簡快字根碼。
\item[三、]容錯字符碼與容錯字根碼\\
使用者可能犯錯而產生的碼。
\end{enumerate}

為了能達到簡快碼及容錯碼,編碼資訊採用以下格式:
\listXML\begin{lstlisting}
<字符 名稱="丈" 註記="U+4E08">
	<組字 運算="龜" 類型="簡快">
		<編碼資訊 字符碼="是" 獨體編碼="qx"/>
	</組字>
</字符>
\end{lstlisting}

其中,類型可以為:``標準''、``容錯''及``簡快'',如果沒有指定的話,預設為``標準''。
而編碼資訊則可以指定``字符碼="是"'',代表是用於字符碼。或加上``字根碼="是"'',代表用於字根碼。
兩者都未指定,表示兩者皆是。

