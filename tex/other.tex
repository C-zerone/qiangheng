\chapter{其它事項}
字頻資訊放於 qhdata/frequency/CJK.xml 。
這是由 SCIM 的廣東拼音字碼表所擷取出來的。


用辭定義:
\begin{enumerate}
\item[一、]字符碼\\
當此部件為一字符時所用的碼。
對一些輸入法而言,又稱作``簡快碼''。
如:嘸蝦米的`一'編碼為 E 但為簡碼,倉頡的`冖‘編碼為 戈弓(IN) 。
\item[二、]字根碼\\
當此部件為一字根時所用的碼。
對一些輸入法而言,又稱作``簡字根''。
如:嘸蝦米的`一'編碼為 E 且補碼為 E,倉頡的`冖‘編碼為 月(B) 。
\end{enumerate}

\begin{enumerate}
\item[一、]標準碼\\
依據輸入法標準而生的碼,且一字一碼。
\item[二、]簡快碼\\
為了加快輸入速度,可能會依字頻而給予較簡編碼產生的碼。 
\item[三、]容錯碼\\
使用者可能犯錯而產生的碼,如選用錯誤字形,寫錯字。
\end{enumerate}

