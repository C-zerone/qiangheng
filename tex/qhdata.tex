\chapter{資料格式}

一個字若是由其它組件所組合,那麼要描述這樣的訊息,
最主要需要描述的資訊有其所組成的``部件''以及這些部件的``結合方式''。
以``曉''這個字為例,其可以分解成``日''和``堯'',而其結合方式為左右結合。

為方便說明,使用``運算''一術語來稱謂``結合方式''。
並且,再把運算分兩種:一為內建的運算,一為``範本''。
所謂的範本,即為用來描述一種模式。
可以發現,很多字是具有類似的模式,
如:``品''、``晶''、``犇''都是三個相同的部件組成品字形。
又如:``贏''、``嬴''、``羸''都是類似結構。
範本定義的模式是由內建運算所建立起來的。

\section{運算與範本}
本計劃所定義及內建的運算有:
\begin{enumerate}
\item[一、]`龜':標示此部件無法被分解。如:木
\item[二、]`爲':此部件等同另一部件。如:釒即為金。
\item[三、]`龍':由多個部件結合成一個合體組件,沒有一定方向性。如:畞
\item[四、]`東':由多個部件結合成一個獨體組件。如:東
\item[五、]`蚕':由多個部件呈現``縱向''的結合。如:想、臱。
\item[六、]`鴻':由多個部件呈現``橫向''的結合。如:相、湘。
\item[七、]`回':由多個部件呈現``包含''的結合。如:國、困。
\item[八、]`錯':未使用。
\end{enumerate}

此外,對於漢字的最常見的幾種模式,在此列表如下:
\begin{enumerate}
\item[一、]`範好':兩個部件左右結合。
\item[二、]`範志':兩個部件上下結合。
\item[三、]`範湘':三個部件左右結合。
\item[四、]`範算':三個部件上下結合。
\item[五、]`範膷':四個部件左右結合。
\item[六、]`範纂':四個部件上下結合。
\item[七、]`範舝':五個部件上下結合。
\end{enumerate}

對於漢字的最常見的幾種重複模式,在此列表如下:
\begin{enumerate}
\item[一、]`範林':兩個相同的部件左右結合。
\item[二、]`範圭':兩個相同的部件上下結合。
\item[三、]`範㴇':三個相同的部件左右結合。
\item[四、]`範鑫':三個相同的部件成品字排列。
\item[五、]`範燚':四個相同部件成田字排列。
\end{enumerate}

一些字有類似的複雜結構,如:
\begin{enumerate}
\item[一、]`範贏':類似``贏''的結構,如:``嬴''、``羸''。
\item[二、]`範微':類似``微''的結構,如:``微''、``徽''。
\item[三、]`範衍':包含於``行''內的結構,如:``衍''、``街''。
\item[四、]`範衷':包含於``衣''內的結構,如:``衷''、``裏''。
\end{enumerate}


\section{格式}
本計劃中描述字形結構的資料採用 XML 。

在此,以``曉''字來為資料格式做說明:
\listXML\begin{lstlisting}
<字符 名稱="曉" 註記="U+66C9">
	<組字 運算="範好">
		<字根 置換="日"/>
		<字根 置換="堯"/>
	</組字>
</字符>
\end{lstlisting}

主要分三部分來說明:
\begin{enumerate}
\item[一、]字符:即將要描述的字符。
包含兩個屬性,``名稱''用來描述目標字等。
``註記''則只是附帶說明。
\item[二、]描述:即要描述的結構。
\item[三、]部件:所指向的部件。
``置換''用來描述所要替換的字符。
\end{enumerate}



此外,``龜''運算比較特別,從定義來說,此運算說明了此字無法被分解。
因此,用此運算來定義編碼資訊。
如羊在倉頡中的編碼為:
\listXML\begin{lstlisting}
<字符 名稱="羊" 註記="U+7F8A">
	<組字 運算="龜">
		<編碼資訊 資訊表示式="tq"/>
	</組字>
</字符>
\end{lstlisting}

而羊在鄭碼的編碼則為
\listXML\begin{lstlisting}
<字符 名稱="羊" 註記="U+7F8A">
	<組字 運算="龜">
		<編碼資訊 資訊表示式="uc"/>
	</組字>
</字符>
\end{lstlisting}


\section{拆解步驟}
在此以之前提到的``筆劃''來做說明。``曉''字可拆解成``日''和``堯''。
因此,要計算``曉''的筆劃數,可以由``日''的筆劃數加上``堯''的筆劃數。
然而,要計算``日''的筆劃數,則要一筆劃一筆劃地計數。
也就是,我們可以分成兩種部件,一種是可拆解的,另一種是不可拆解的。

對於合體字而言,其描述通常是與輸入法相獨立的。
如``曉''字可拆成``日''和``堯'',這點對任何輸入法都是一樣的。
而對於獨體字而言,其資訊則跟輸入法有關。
如:對計算筆劃而言,``日''為四劃,對倉頡而言,日的拆碼為日(a),
而對鄭碼而言,則為 k 。

在區分獨體字和合體字後。
上面所說,合體字的描述通常是與輸入法相獨立的,這點是不完全對的。
不同的輸入法有時也會對一些字有不同的拆解看法。
如``亘''這個字,在倉頡中的看法是由``一''、``日''、``一''所組成,
而在鄭碼中則視為``二''、``日''所組成。

縱合上述,將拆碼過程分為三步驟:
一、通用型拆碼,對大部分的輸入法都適用。將字拆成組件。
二、將組件拆成字根。
三、對組件編碼。

此外,不同輸入法會採用不同的字形。
譬如,行列、大易、嘸蝦米、倉頡以繁體字為主,而鄭碼以簡體字為主
如,``堩'',行列、大易、嘸蝦米、倉頡視為``土''及``恆'',
而鄭碼則視為``土''及``恒''。


\section{目錄結構}
目錄結構為:
\begin{lstlisting}
qhdata/
	component/
	config/
	main/
	radix/
	style/
	template/
\end{lstlisting}

其中,template/ 是用來存放範本的。config/ 則是存放設定檔。
main/ 則存放最通用的拆解方式。
component/ 則存放部件轉換成字根的描述。
radix/ 則存放字根轉成編碼的描述。
style/ 則存放不同字形。


